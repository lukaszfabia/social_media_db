\documentclass{article}

% Language setting
% Replace `english' with e.g. `spanish' to change the document language
\usepackage[utf8]{inputenc}
\usepackage{polski}

% Set page size and margins
% Replace `letterpaper' with `a4paper' for UK/EU standard size
\usepackage[letterpaper,top=2cm,bottom=2cm,left=3cm,right=3cm,marginparwidth=1.75cm]{geometry}

\usepackage{float}

% Useful packages
\usepackage{amsmath}
\usepackage{graphicx}
\usepackage[colorlinks=true, allcolors=blue]{hyperref}

\title{Baza danych medium społecznościowego}
\author{Łukasz Fabia 272724 \\ Mikołaj Kubś 272662 \\ Martyna Łopianiak 272682 \\ Piotr Schubert 272659 \\ Projektowanie baz danych wt 18:55}


\begin{document}
\maketitle

\tableofcontents

\section{Etap 1: Faza konceptualna}

\subsection{Analiza świata rzeczywistego}

\subsubsection{Streszczenie - Zarys wymagań projektu}

Celem projektu jest stworzenie bazy danych do obsługi medium społecznościowego. Ma ona przechowywać informacje o użytkownikach, ich treściach, relacjach i aktywnościach. Baza powinna być zaprojektowana w sposób wydajny i skalowalny.

\subsubsection{Potrzeby informacyjne}
\begin{itemize}
    \item Rejestracja i logowanie użytkowników z różnymi poziomami dostępu.
    \item Publikowanie i interakcje z treściami użytkowników (posty, polubienia, komentarze).
    \item Zarządzanie relacjami społecznymi (znajomości).
    \item Przechowywanie wiadomości prywatnych i historii aktywności.
    \item Tworzenie konwersacji z innymi użytkownikami
    \item Reportowanie postów z nieodpowiednimi treśćmi
    \item Tworzenie i zarządzanie stronami organizacji, firm, fanpage itd.
    \item Tworzenie i zarządzanie wydarzeniami
\end{itemize}

\subsubsection{Czynności, wyszukiwania}

\begin{itemize}
    \item Wyszukiwanie użytkowników.
    \item Wyszukiwanie treści po hasztagach lub słowach kluczowych.
    \item Filtrowanie aktywności użytkownika, np. przeglądanie polubień i komentarzy.
    \item Wyszukiwanie relacji (np. znajomi użytkownika, osoby obserwujące daną osobę).
    \item Dodawanie innych użytkowników do znajomych i interakcja z treściami - dodawanie komentarzy i reakcji
    \item Tworzenie postów, wydarzeń, grup
    \item Konwersacja grupowa, pisanie wiadomości
\end{itemize}

\subsubsection{Cele projektu}

\begin{tabular}{@{} l p{12cm} @{}}
    \textbf{S (Specific)}: & Zaprojektowanie bazy danych dla medium społecznościowego. \\ \\
    \textbf{M (Measurable)}: & Baza musi być wydajna, tzn. musi być w stanie obsługiwać dużą ilość użytkowników, co najmniej 20 000. \\ \\
    \textbf{A (Achievable)}: & Projekt zostanie zrealizowany przy użyciu PostgreSQL. Do stworzenia struktury tabel wykorzystany zostanie mechanizm ORM (Object Relational Mapping). Na koniec baza zostanie wypełniona danymi, aby przetestować jej wydajność. \\ \\
    \textbf{R (Relevant)}: & Przechowywanie profili użytkowników oraz interakcje między nimi są kluczowe dla funkcjonowania medium społecznościowego. \\ \\
    \textbf{T (Time-bound)}: & Praca nad projektem powinna zająć 2 miesiące.
\end{tabular}

\subsubsection{Zakres projektu}

\begin{tabular}{@{} l p{10cm} @{}}
    \textbf{Multimedia}: & W bazie przechowywane będą wyłącznie linki do plików na zewnętrznym serwerze. \\
    \textbf{Obsługa haseł}: & Wszystkie hasła w bazie będą hashowane.
\end{tabular}


\subsection{Wymagania funkcjonalne}

\begin{abstract}
    Użytkownikom przypisany jest jeden z tych poziomów dostępu: admin, user lub guest.
\end{abstract}

\textbf{Guest(Gość)}

\begin{itemize}
    \item Może przeglądać wybrane dane.
\end{itemize}

\textbf{Admin}

\begin{itemize}
    \item Może przeglądać, edytować, usuwać, dodawać i przeglądać wszystkie treści, zrządza bazą i nadaje uprawnienia
\end{itemize}

\textbf{User(Użytkownik)}

\begin{itemize}
    \item System umożliwia rejestrację oraz logowanie.
    \item Rejestracja wymaga imienia, nazwiska, daty urodzenia, hasła oraz maila.
    \item Logowanie wymaga maila i hasła.
\end{itemize}


\subsection{ERD}

\begin{figure}[H]
    \centering
    \includegraphics[width=\linewidth]{images/Blank diagram.png}
    \caption{Diagram obiektowo-relacyjny}
    \label{fig:erd}
\end{figure}

\section{Etap 2: Faza logiczna}

Do tabel zostały dodane atrybuty, tabele \textbf{Page} oraz \textbf{User} zostały uogólnione przez \textbf{Author}, który bierze udział w innych czynnościach. Baza została także sprowadzona do \textit{III postaci normalniej}, przez co wydzielono klika nowych tabel. 

\begin{figure}[H]
    \centering
    \includegraphics[width=\linewidth]{images/rel.png}
    \caption{Diagram relacji}
    \label{fig:rel}
\end{figure}

\section{Etap 3: Faza fizyczna}

Encje, które uległy zmianie:

\begin{itemize}
    \item Rozbicie lokalizacji na pomniejsze tabele
    \item Zamiana na enumy: typu autora oraz statusu zaproszenia do znajomych
    \item Dodanie do encji Conversation pole Members, które przechowuje id użytkowników biorących udział w konwersacji
\end{itemize}


Wykorzystano instrukcję CHECK do potwierdzenia poprawności danych w paru encjach. Przykładowo:
\begin{itemize}
    \item 'A' nie moze być przyjacielem 'A'
    \item 'A' nie moze wysłać zaproszenia do znajomych do 'A'
    \item Czas rozpoczęcia wydarzenia musi być wcześniejszy niż czas zakończenia wydarzenia
    \item Grupa musi mieć od 1 do 10000000 członków.
\end{itemize}

Do stworzenia struktury bazy wykorzystano mechanizm ORM - (\href{https://gorm.io/}{gorm}). Sama baza wymagała dopracowania jeśli chodzi o enumeracje oraz reguły usuwania już bezpośrednio w systemie PostgreSQL (pgadmin). 

\textbf{Usuwanie}: Każdy model posiada pole \textit{DeletedAt} z indeksem. Podczas usuwania danego wiersza pole \textit{DeletedAt} jest ustawiane na znacznik czasu (timestamp) wskazujący moment, w którym dane zostały usunięte. Dzięki temu baza obsługuje soft deleting. Oznacza to, że gdy użytkownik usunie swoje konto, będzie można je przywrócić. Taką operację obsługują na przykład serwisy takie jak Facebook.

\begin{figure}[H]
    \centering
    \includegraphics[width=\linewidth]{images/postgres_diagram.png}
    \caption{Diagram relacji z PostgreSQL}
    \label{fig:postgres}
\end{figure}

\end{document}
